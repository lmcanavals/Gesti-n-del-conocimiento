\documentclass[conference]{IEEEtran}
\usepackage{cite}
\usepackage{amsmath,amssymb,amsfonts}
\usepackage{algorithmic}
\usepackage{graphicx}
\usepackage{textcomp}
\usepackage{xcolor}
\def\BibTeX{{\rm B\kern-.05em{\sc i\kern-.025em b}\kern-.08em
    T\kern-.1667em\lower.7ex\hbox{E}\kern-.125emX}}
\begin{document}

\title{Servicios Cognitivos y su Impacto en la Gestión del Conocimiento}

\author{\IEEEauthorblockN{Danny Goicochea Mendo}
\IEEEauthorblockA{\textit{Escuela de Postgrado} \\
\textit{UPAO}\\
Trujillo, Perú \\
dgoicocheam1@upao.edu.pe}
\and
\IEEEauthorblockN{Borys Rojas Jaramillo}
\IEEEauthorblockA{\textit{Escuela de Postgrado} \\
\textit{UPAO}\\
Trujillo, Perú \\
brojasj1@upao.pe}
\and
\IEEEauthorblockN{Enrique Aquino Rojas}
\IEEEauthorblockA{\textit{Escuela de Postgrado} \\
\textit{UPAO}\\
Trujillo, Perú \\
daquinor@upao.pe}
\and
\IEEEauthorblockN{Luis Canaval Sánchez}
\IEEEauthorblockA{\textit{Escuela de Postgrado} \\
\textit{UPAO}\\
Trujillo, Perú \\
lcanavals@upao.edu.pe}
}

\maketitle

\begin{abstract}
Este artículo tiene como objetivo comprender los servicios cognitivos en el ámbito tecnológico y empresarial y su impacto en la gestión del conocimiento para fines académicos y sociales. Los resultados nos permiten concluir que estos servicios cognitivos, no solo están siendo explotados a nivel tecnológico por las grandes corporaciones a nivel comercial, sino también existen servicios open source al alcance de todos y cuyo impacto en la gestión del conocimiento se traduce en creación de agentes de cambio no solo por el uso del conocimiento como información en una organización, sino que puede ser explotada para llegar a un nivel de generación de nuevo conocimiento que genere un valor competitivo ya sea a nivel de procesos, tiempos de respuesta en acceso  a la información y su valor intrínseco en la toma de decisiones.
\end{abstract}

\begin{IEEEkeywords}
cognitive services, knowledge management
\end{IEEEkeywords}

\section{Introduction}
La introducción va acá
\section{Marco Teórico}

\subsection{Servicios Cognitivos}

Los servicios cognitivos (Cognitive Services) ponen la inteligencia artificial al alcance de todos los profesionales sin que para utilizarla sea necesario contar con experiencia aprendizaje máquina (Machine Learning) a nivel técnico. Basta con una llamada a una API para incorporar la capacidad de ver, escuchar, hablar, buscar, comprender y potenciar la toma de decisiones en las aplicaciones \cite{Shwartz2019}.

Los servicios cognitivos permiten dotar a los sistemas informáticos de diversa índole, de las capacidades que tenemos los humanos de percibir y procesar aquello que ha sido percibido, apoyándose en las capacidades que proporciona el aprendizaje máquina. La tecnología cuenta con la ventaja de que la tecnología no sufre de errores por cansancio o aburrimiento ante tareas repetitivas, y por lo tanto son capaces de realizar muchos trabajos en mejores condiciones que su equivalente humano. Gracias a los servicios cognitivos es posible dotar cualquier producto de inteligencia artificial sin contar con un equipo de expertos en aspectos técnicos de alta complejidad.

\subsection{Azure Cognitive Service}

Los servicios cognitivos son una serie de servicios en forma de API REST creadas por Microsoft que nos facilitan el uso de la Inteligencia artificial de una manera fácil y directa a todas nuestras aplicaciones \cite{Brusakova2020}.

Estos servicios se dividen en cinco grandes categorías que son:

\subsubsection{Visión} Las APIs de esta categoría nos ayudan a identificar cosas tales como objetos o caras (reconocimiento facial) dentro de imágenes. También nos permiten identificar emociones (contento, enfadado, disgustados, etc.) tanto en imágenes como en videos con caras.

\subsubsection{Voz} Estas APIs de voz nos permiten hacer cosas tales como convertir el texto en voz y viceversa. Por ejemplo, también podemos usar en concreto la API Speaker Recognition API para identificar y verificar voces y usarlo dentro de nuestros en sistemas de autenticación (reconocer la voz de una persona para darle acceso a una aplicación, por ejemplo).

\subsubsection{Conocimiento}: Estas APIs nos permiten por ejemplo recomendar productos a clientes dependiendo de la actividad pasada del usuario, en el caso por ejemplo de que tuviéramos una tienda online. También nos pueden ayudar a extraer información relevante dentro de textos.

\subsubsection{Búsqueda} Estas APIs nos proporcionan capacidades de búsqueda dentro de nuestras aplicaciones usando como motor Bing.com. Estas APIs nos pueden ayudar en cosas como la búsqueda de imágenes, noticias, videos y páginas web.

\subsubsection{Lenguaje} Estas APIs nos pueden ayudar a realizar cosas como corrección gramatical y ortográfica de textos. También podemos usarlas para extraer análisis del sentimiento de textos o incluso cosas tales como detectar texto dentro de escritos a mano \cite{Ogiela2018}.

Una de las API más importantes es LUIS (Language Understanding Intelligent Service), que nos permite incluir en nuestras aplicaciones en las que el usuario se comunica con nosotros a través de una conversación, poder entender qué acciones quiere realizar el usuario en nuestras aplicaciones. Decir que esta API es el compañero perfecto en el desarrollo de chatbots, en el que el medio de comunicarse el usuario con nuestra aplicación es a través de conversaciones.


\input{03-casos}
\section{Gestión del conocimiento}
\section{Propuestas}
Por lo observado en los casos estudiados, el uso de servicios cognitivos en el mundo empresarial se desarrolla de manera robusta en el contexto tradicional, aplicación directa de las tecnologías cognitivas basadas en algoritmos de machine learning y data mining que han alcanzado un nivel de madurez y robustez de acuerdo a los altos estándares característicos de la industria. Sin embargo, existen áreas de desarrollo para los servicios cognitivos que, en conjunto con otras herramientas pueden presentar la oportunidad de aplicaciones en un nuevo nivel de inovación y disrupción \cite{Majarres}.
El uso de herramientas basadas en GTP-3, Transformers \cite{Vaswani2017}, agentes para tareas complejas, los cuales son servicios que podemos argumentar se encuentran en sala de espera para convertirse en la siguiente capa de servicios cognitivos, permiten imaginar las posibles aplicaciones en las cuales, los sistemas de acceso al conocimiento no solo cumplan un rol facilitador, sino un rol más activo e incluso un rol de producción de conocimiento. Por ejemplo al combinar ontologías de distintos aspectos o de distintas áreas de negocio, podría ser posible la inferencia de nuevo conocimiento. Del mismo modo, la combinación de transformadores y agentes para tareas complejas se presentan como la posibilidad de facilitar enormemente las tareas de profesionales que realizan tareas de un nivel de complejidad asociado a un nivel de conocimiento, hasta el momento considerado fuera del alcance de los sistemas conversacionales.
\section{Conclusiones}
El presente trabajo partió con la premisa de investigar el impacto de los servicios cognitivos en la gestión del conocimiento, para ello, primero definimos qué son dichos servicios cognitivos, desde un punto de vista teórico, es decir las áreas de la inteligencia artificial asociados a ellos; pero también desde un punto de vista comercial, identificando a los principales actores que ofrecen éstos servicios como un producto que ya tiene un nivel de adopción alto, como Microsoft Azure Cognitive Services e IBM Watson.
Encontramos que las aplicaciones que se le da a ésta tecnología en las empresas, especialmente de la región, tiene un profundo impacto en el acceso al conocimiento de las organizaciones, pues, uno de los principales servicios es el de búsqueda, permitiendo simplificar y minimizar el esfuerzo para clasificar las enormes moles de conocimiento, que tradicionalmente requiere hacerse a mano. Con los servicios cognitivos, se puede organizar dicha información y ponerla a disposición usando servicios de búsqueda simples de usar, pero tremendamente sofisticados internamente.
Finalmente esbozamos un vistazo a un futuro, de momento hipotético, en el que los servicios cognitivos, junto con otras tecnologías como ontologías, agentes para tareas complejas, entre otras, podrían ser usadas en conjunto para construir agentes que no solo utilicen el conocimiento de una organización, no solamente lo hagan disponible para las personas, sino que también produzca nuevo conocimiento e incluso se puedan desempeñar con mínima supervisión humana en tareas críticas de apoyo a la toma de decisiones. 

\bibliographystyle{ieeetr}
\bibliography{bibliography.bib}

\end{document}