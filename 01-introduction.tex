\section{Introducción}

La gestión del conocimiento es una disciplina en proceso de adopción por un gran numero de
empresas, organizaciones e instituciones, cuyo proceso es arduo y demanda una gran inversión de
recursos. Por otro lado, los sistemas cognitivos, como un producto orientado a simplificar la
aplicación de poderosos conceptos de inteligencia artificial para el procesamiento de datos, se
presentan como el siguiente paso en la evolución natural de la industria en general. Servicios
como Azure Cognitive Search o IBM Watson Discovery \cite{Tadejko2020}, ganan cada vez
más clientes con el pasar de los recientes años. El presente estudio busca resaltar el impacto
que ha tenido la adopción de servicios cognitivos de diversa índole, en las distintas etapas de
la gestión del conocimiento como un proceso continuo. Investigamos casos de aplicación de
servicios cognitivos en empresas y resaltamos el impacto positivo obtenido.

En la sección II describimos conceptos importantes para el mejor entendimiento del presente
estudio, en la sección III presentamos algunos casos de aplicación donde destacamos el impacto
logrado, en la sección IV presentamos datos del uso de servicios cognitivos para la gestión del
conocimiento, en la sección V discutimos posibles nichos de aplicación de servicios cognitivos
como parte integral de la gestión del conocimiento, finalmente en la sección VI presentamos
nuestras conclusiones \cite{Ogiela2018}.