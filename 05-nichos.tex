\section{Propuestas}
Por lo observado en los casos estudiados, el uso de servicios cognitivos en el mundo empresarial se desarrolla de manera robusta en el contexto tradicional, aplicación directa de las tecnologías cognitivas basadas en algoritmos de machine learning y data mining que han alcanzado un nivel de madurez y robustez de acuerdo a los altos estándares característicos de la industria. Sin embargo, existen áreas de desarrollo para los servicios cognitivos que, en conjunto con otras herramientas pueden presentar la oportunidad de aplicaciones en un nuevo nivel de inovación y disrupción \cite{Majarres}.
El uso de herramientas basadas en GTP-3, Transformers \cite{Vaswani2017}, agentes para tareas complejas, los cuales son servicios que podemos argumentar se encuentran en sala de espera para convertirse en la siguiente capa de servicios cognitivos, permiten imaginar las posibles aplicaciones en las cuales, los sistemas de acceso al conocimiento no solo cumplan un rol facilitador, sino un rol más activo e incluso un rol de producción de conocimiento. Por ejemplo al combinar ontologías de distintos aspectos o de distintas áreas de negocio, podría ser posible la inferencia de nuevo conocimiento. Del mismo modo, la combinación de transformadores y agentes para tareas complejas se presentan como la posibilidad de facilitar enormemente las tareas de profesionales que realizan tareas de un nivel de complejidad asociado a un nivel de conocimiento, hasta el momento considerado fuera del alcance de los sistemas conversacionales.