\section{Conclusiones}
El presente trabajo partió con la premisa de investigar el impacto de los servicios cognitivos en la gestión del conocimiento, para ello, primero definimos qué son dichos servicios cognitivos, desde un punto de vista teórico, es decir las áreas de la inteligencia artificial asociados a ellos; pero también desde un punto de vista comercial, identificando a los principales actores que ofrecen éstos servicios como un producto que ya tiene un nivel de adopción alto, como Microsoft Azure Cognitive Services e IBM Watson.
Encontramos que las aplicaciones que se le da a ésta tecnología en las empresas, especialmente de la región, tiene un profundo impacto en el acceso al conocimiento de las organizaciones, pues, uno de los principales servicios es el de búsqueda, permitiendo simplificar y minimizar el esfuerzo para clasificar las enormes moles de conocimiento, que tradicionalmente requiere hacerse a mano. Con los servicios cognitivos, se puede organizar dicha información y ponerla a disposición usando servicios de búsqueda simples de usar, pero tremendamente sofisticados internamente.
Finalmente esbozamos un vistazo a un futuro, de momento hipotético, en el que los servicios cognitivos, junto con otras tecnologías como ontologías, agentes para tareas complejas, entre otras, podrían ser usadas en conjunto para construir agentes que no solo utilicen el conocimiento de una organización, no solamente lo hagan disponible para las personas, sino que también produzca nuevo conocimiento e incluso se puedan desempeñar con mínima supervisión humana en tareas críticas de apoyo a la toma de decisiones. 