\section{Marco Teórico}

\subsection{Servicios Cognitivos}

Los servicios cognitivos (Cognitive Services) ponen la inteligencia artificial al
alcance de todos los profesionales sin que para utilizarla sea necesario contar con
experiencia aprendizaje máquina (Machine Learning) a nivel técnico. Basta con una llamada
a una API para incorporar la capacidad de ver, escuchar, hablar, buscar, comprender y
potenciar la toma de decisiones en las aplicaciones.

Los servicios cognitivos permiten dotar a los sistemas informáticos de diversa índole, de las
capacidades que tenemos los humanos de percibir y procesar aquello que ha sido percibido,
apoyándose en las capacidades que proporciona el aprendizaje máquina. La tecnología cuenta con
la ventaja de que la tecnología no sufre de errores por cansancio o aburrimiento ante tareas
repetitivas, y por lo tanto son capaces de realizar muchos trabajos en mejores condiciones que
su equivalente humano. Gracias a los servicios cognitivos es posible dotar cualquier producto
de inteligencia artificial sin contar con un equipo de expertos en aspectos técnicos de alta
complejidad.

\subsection{Azure Cognitive Service}
Los servicios cognitivos son una serie de servicios en forma de API REST creadas por Microsoft que nos facilitan el uso de la Inteligencia artificial de una manera fácil y directa a todas nuestras aplicaciones. %(García Miravet, 2018)

Estos servicios se dividen en cinco grandes categorías que son:

\subsubsection{Visión} Las APIs de esta categoría nos ayudan a identificar cosas tales como
objetos o caras (reconocimiento facial) dentro de imágenes. También nos permiten identificar
emociones (contento, enfadado, disgustados, etc.) tanto en imágenes como en videos con caras.

\subsubsection{Voz} Estas APIs de voz nos permiten hacer cosas tales como convertir el texto
en voz y viceversa. Por ejemplo, también podemos usar en concreto la API Speaker Recognition
API para identificar y verificar voces y usarlo dentro de nuestros en sistemas de autenticación
(reconocer la voz de una persona para darle acceso a una aplicación, por ejemplo).

\subsubsection{Conocimiento}: Estas APIs nos permiten por ejemplo recomendar productos a
clientes dependiendo de la actividad pasada del usuario, en el caso por ejemplo de que
tuviéramos una tienda online. También nos pueden ayudar a extraer información relevante dentro
de textos.

\subsubsection{Búsqueda} Estas APIs nos proporcionan capacidades de búsqueda dentro de nuestras
aplicaciones usando como motor Bing.com. Estas APIs nos pueden ayudar en cosas como la búsqueda
de imágenes, noticias, videos y páginas web.

\subsubsection{Lenguaje} Estas APIs nos pueden ayudar a realizar cosas como corrección
gramatical y ortográfica de textos. También podemos usarlas para extraer análisis del
sentimiento de textos o incluso cosas tales como detectar texto dentro de escritos a mano.

Una de las API más importantes es LUIS (Language Understanding Intelligent Service), que nos permite incluir en nuestras aplicaciones en las que el usuario se comunica con nosotros a través de una conversación, poder entender qué acciones quiere realizar el usuario en nuestras aplicaciones. Decir que esta API es el compañero perfecto en el desarrollo de chatbots, en el que el medio de comunicarse el usuario con nuestra aplicación es a través de conversaciones.

